%%%%%%%%%%%%%%%%%%%%%%%%%%%%%%%%%%%%%%%%%
% fphw Assignment
% LaTeX Template
% Version 1.0 (27/04/2019)
%
% This template originates from:
% https://www.LaTeXTemplates.com
%
% Authors:
% Class by Felipe Portales-Oliva (f.portales.oliva@gmail.com) with template 
% content and modifications by Vel (vel@LaTeXTemplates.com)
%
% Template (this file) License:
% CC BY-NC-SA 3.0 (http://creativecommons.org/licenses/by-nc-sa/3.0/)
%
%%%%%%%%%%%%%%%%%%%%%%%%%%%%%%%%%%%%%%%%%

%----------------------------------------------------------------------------------------
%	PACKAGES AND OTHER DOCUMENT CONFIGURATIONS
%----------------------------------------------------------------------------------------

\documentclass[
	12pt, % Default font size, values between 10pt-12pt are allowed
	%letterpaper, % Uncomment for US letter paper size
	%spanish, % Uncomment for Spanish
	german, % Uncomment for German
]{fphw}

% Template-specific packages

%encoding
%--------------------------------------
\usepackage[utf8]{inputenc}
\usepackage[T1]{fontenc}
%--------------------------------------
 
%German-specific commands
%--------------------------------------
\usepackage[ngerman]{babel}
\usepackage{csquotes}

\usepackage{mathpazo} % Use the Palatino font

\usepackage{graphicx} % Required for including images

\usepackage{booktabs} % Required for better horizontal rules in tables

\usepackage{listings} % Required for insertion of code

\usepackage{enumerate} % To modify the enumerate environment

%----------------------------------------------------------------------------------------
%	ASSIGNMENT INFORMATION
%----------------------------------------------------------------------------------------

\title{Berechenbarkeit - korrekte Argumentation} % Assignment title

\author{Alexandra Maximova} % Student name

\date{25.03.2020} % Due date

\institute{ETH Zurich \\ Lehrdiplom Informatik} % Institute or school name

\class{Fachdidaktik 2 (Berechenbarkeit)} % Course or class name

\professor{Giovanni Serafini, Juraj Hromkovič} % Professor or teacher in charge of the assignment

%----------------------------------------------------------------------------------------

\begin{document}

\maketitle % Output the assignment title, created automatically using the information in the custom commands above

%----------------------------------------------------------------------------------------
%	ASSIGNMENT CONTENT
%----------------------------------------------------------------------------------------

\section*{Frage 3.1 (a)}

\begin{problem}
	Überlegen Sie, wie Ihre Schüler argumentieren könnten, um die Regen-Wiese-Salamander Metapher an ihre Grenzen zu bringen.
\end{problem}

%------------------------------------------------

\subsection*{Lösung}

\begin{enumerate}
\item Kann es nicht sein, dass der eine Salamander einen besonders schlechten Tag hat, und sich nicht mal über eine nasse Wiese freut?
\item Kann es nicht sein, dass Salamander eigentlich den Regen hassen und nur nasse Wiese mögen? Dann ein unendlich langer Regen macht sie sicher nicht glücklich.
\item Kann es nicht sein, dass es so viele Salamander auf die Wiese kommen, dass kein Tropfen auf die Wiese kommt? Dann wäre die Wiese auch nach dem Regen trocken.
\end{enumerate}

%----------------------------------------------------------------------------------------

\section*{Frage 3.1 (b)}

\begin{problem}
	Schlagen Sie eine möglichst neue, kreative, eigene Metapher vor. Sie soll zunächst zwei und dann drei Aussagen beinhalten.
\end{problem}

%------------------------------------------------

\subsection*{Lösung}

\begin{enumerate}
\item "Wenn es 00 Minuten ist, dann läuten die Kirchenglocken" -- "Wenn die Kirchenglocken läuten, schrecken die Vögel auf" \\
Warum es gut ist: Kirchenglocken läuten auch in anderen Fällen. \\
Warum es schlecht ist: Läuten Kirchenglocken auch nachts? Haben alle Vögel eine Kirche in der Nähe? Ich kann auf dem Everest um Punkt 1 stehen und keine Vögel, die aufschrecken, beobachten. Haben sich die Vögel immer noch nicht daran gewöhnt, einmal pro Stunde aufgescheucht zu werden?
\item "Wenn man etwas esssen möchte, dann muss jemand zuerst Geschirr spülen" \\
Warum es schlecht ist: Haben wir denn nicht genug Geschirr?
\end{enumerate}

\end{document}
