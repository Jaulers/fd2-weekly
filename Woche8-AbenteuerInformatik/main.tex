%%%%%%%%%%%%%%%%%%%%%%%%%%%%%%%%%%%%%%%%%
% fphw Assignment
% LaTeX Template
% Version 1.0 (27/04/2019)
%
% This template originates from:
% https://www.LaTeXTemplates.com
%
% Authors:
% Class by Felipe Portales-Oliva (f.portales.oliva@gmail.com) with template 
% content and modifications by Vel (vel@LaTeXTemplates.com)
%
% Template (this file) License:
% CC BY-NC-SA 3.0 (http://creativecommons.org/licenses/by-nc-sa/3.0/)
%
%%%%%%%%%%%%%%%%%%%%%%%%%%%%%%%%%%%%%%%%%

%----------------------------------------------------------------------------------------
%	PACKAGES AND OTHER DOCUMENT CONFIGURATIONS
%----------------------------------------------------------------------------------------

\documentclass[
	12pt, % Default font size, values between 10pt-12pt are allowed
	%letterpaper, % Uncomment for US letter paper size
	%spanish, % Uncomment for Spanish
	german, % Uncomment for German
]{fphw}

% Template-specific packages

%encoding
%--------------------------------------
\usepackage[utf8]{inputenc}
\usepackage[T1]{fontenc}
%--------------------------------------
 
%German-specific commands
%--------------------------------------
\usepackage[ngerman]{babel}
\usepackage{csquotes}

\usepackage{mathpazo} % Use the Palatino font

\usepackage{graphicx} % Required for including images

\usepackage{booktabs} % Required for better horizontal rules in tables

\usepackage{listings} % Required for insertion of code

\usepackage{enumerate} % To modify the enumerate environment

%----------------------------------------------------------------------------------------
%	ASSIGNMENT INFORMATION
%----------------------------------------------------------------------------------------

\title{Wie die Informatik zum Abenteuer wird} % Assignment title

\author{Alexandra Maximova} % Student name

\date{22.04.2020} % Due date

\institute{ETH Zurich \\ Lehrdiplom Informatik} % Institute or school name

\class{Fachdidaktik 2 (Abenteuer Informatik)} % Course or class name

\professor{Giovanni Serafini} % Professor or teacher in charge of the assignment

%----------------------------------------------------------------------------------------

\begin{document}

\maketitle % Output the assignment title, created automatically using the information in the custom commands above

%----------------------------------------------------------------------------------------
%	ASSIGNMENT CONTENT
%----------------------------------------------------------------------------------------

\section*{Spielchen gefällig?}

\begin{problem}
	Formuliere eine Leitidee zum Kapitel 16 (Spielchen gefällig?) vom Buch ''Abenteuer Informatik'' von Jens Gallenbacher.
\end{problem}

%------------------------------------------------

\subsection*{Leitidee}

Menschen sind nicht allwissend und leben in einem gerichteten Zeitstrahl. Sie haben keinen Zugang zu Informationen aus der Zukunft, und trotzdem müssen sie Entscheidungen treffen und Probleme lösen. Probleme, die hier und jetzt gelöst werden müssen, während die Informationen für eine optimale Lösung erst später zur Verfügung stehen werden, treten sehr häufig auf. Zum Beispiel, wenn ein Mensch eine Saisonkarte fürs Freibad kauft, weiss er nicht, ob er diese auch tatsächlich nutzen wird. Aber er kann nicht bis September warten, sonnige Tage zählen und erst dann zurück in die Zeit reisen, um in Juni eine Saisonkarte zu kaufen.

Deswegen ist es wichtig, dass die SuS mit solchen Problemen konfrontiert werden und sehen, dass man mit der richtigen Formalisierung und Modellierung sogar in diesen scheinbar hoffnungslosen Fälle relativ gute Entscheidungen treffen kann.


%----------------------------------------------------------------------------------------

\end{document}
